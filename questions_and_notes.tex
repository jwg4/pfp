\documentclass{article}

\usepackage{urls}

\begin{document}
\section{Yahoo Finance app}
\section{Quantlib app}
\emph{
QuantLib is an open-source C++ library of financial pricing functions. It has an interface to various other applications (eg Excel) and languages (eg Python).
We use QuantLib a lot and I would like to see if you can get your head around it a bit.
I have previously asked candidates to build it in MS Visual Studio 2008 (there are guides for this on the web), but actually the core C++ library and the SWIG interface to Python can be apt-get installed in an Ubuntu distribution.
Here is the exercise: set up a web app that uses Python/QuantLib to do some kind of asset pricing.
Most of our work is focussed on ‘fixed income’ products at the moment, which generally means ‘swaps’ and ‘swaptions’, so I would suggest that you try to price a user-specified swap with the discount factors coming from a simple user-specified yield curve.
If you were doing this within Excel (the QuantLib developers have also made it possible to expose a lot of the funtions to Excel) then you would use functions likeqlSchedule, qlSwap,qlInstrumentNPV to create the Swap object and then price it.
To price swaps you need to first build a yield curve; there is example code on the web, and also in the packages I believe.
I understand that you might not know anything at all about interest rates, derivatives, swaps, but do not worry! I am really just keen to see if you can wrap some example QuantLib code in a good-looking web application.}


\begin{enumerate}
\item
I have created a github repository for all the code in the application. That means you can see all the code, but also follow the progress through the different commits at \url{https://github.com/jwg4/pfp}. 
\item
I set the system up on a debian installation on a raspberry pi which I use as a test server. Debian packages are often a little older than the corresponding ubuntu ones, and Raspberry Pi distro \url{raspbian} is probably a little behind that, so some of the tools used might be a long way off what's currently the most recent.
\item 
I used the \url{nginx} web server, the \url{uWSGI} app server and the \url{django} framework.
\item
I check all the config files in \url{\etc} on my server into a git repository which is also on github at \url{https://github.com/jwg4/pfp}. The idea is that this makes it easy to rollback changes, to backup important configs, and to replicate the system onto another machine. I included this repo as a `submodule' of the git repo - this means that when I had to make changes to \url{\etc}, I could check them into the \url{pfp} repository. So I have a record of how I had to configure the web server and the other services to get Django up and running.  
\item
I have added all the package installations that I had to do, and a couple of other things like starting \url{nginx} and \url{uWSGI} to a script called \url{setup.bash}. The idea is that it should be possible to setup the same system on a clean install of \url{raspbian} just by running this script and replicating the changes checked into \url{jwg-etc}. (I haven't tested this!)   
\end{enumerate}
\end{document}